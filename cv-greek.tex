%%%%%%%%%%%%%%%%%%%%%%%%%%%%%%%%%%%%%%%%%
% Medium Length Professional CV
% LaTeX Template
% Version 2.0 (8/5/13)
%
% This template has been downloaded from:
% http://www.LaTeXTemplates.com
%
% Original author:
% Trey Hunner (http://www.treyhunner.com/)
%
% Important note:
% This template requires the resume.cls file to be in the same directory as the
% .tex file. The resume.cls file provides the resume style used for structuring the
% document.
%
%%%%%%%%%%%%%%%%%%%%%%%%%%%%%%%%%%%%%%%%%

%----------------------------------------------------------------------------------------
%	PACKAGES AND OTHER DOCUMENT CONFIGURATIONS
%----------------------------------------------------------------------------------------

\documentclass{resume} % Use the custom resume.cls style

\usepackage[left=0.75in,top=0.6in,right=0.75in,bottom=0.6in]{geometry} % Document margins
\usepackage{hyperref}
\usepackage{longtable}
\usepackage{xspace}

\usepackage[utf8]{inputenc}
\usepackage[greek,english]{babel}
\usepackage{lmodern}

\name{\textgreek{Λεωνίδας Λαμπρόπουλος}} % Your name
%\address{123 Broadway \\ City, State 12345} % Your address
%\address{123 Pleasant Lane \\ City, State 12345} % Your secondary addess (optional)
%\address{(000)~$\cdot$~111~$\cdot$~1111 \\ john@smith.com} % Your phone number and email
\address{Website: \href{https://lemonidas.github.io}{lemonidas.github.io}}
\address{e-mail: leonidas@umd.edu}

\begin{document}

%----------------------------------------------------------------------------------------
%	EDUCATION SECTION
%----------------------------------------------------------------------------------------

\newcommand{\edu}[4]{
  {\bf #3} \hfill {\em #2}\\
  {#1}\\
  #4\\
}

\begin{rSection}{\textgreek{Εκπαίδευση και Εργασία}}

\edu{\textgreek{Επίκουρος Καθηγητής}}{\textgreek{Ιούλιος 2020 - τώρα}}
    {University of Maryland, College Park}
    {\textgreek{Τμήμα Επιστήμης Υπολογιστών}}
  
\edu{\textgreek{Μεταδιδακτορικός Ερευνητής Επιστήμης Υπολογιστών}}{2018 - 2020}
    {University of Maryland, College Park \textgreek{και} University of Pennsylvania}{
      \textgreek{Μεταδιδακτορική Υποτροφία} Victor Basili\\
      \textgreek{Ομάδα Γλωσσών Προγραμματισμού} (PLUM \textgreek{και} PL Club)\\
      \textgreek{Επιβλέψη: }Michael Hicks, Benjamin C. Pierce
    }

\edu{\textgreek{Διδακτορικό στην Επιστήμη Υπολογιστών}}{2012 - 2018}
    {University of Pennsylvania}{
      %      GPA: 3.99\\
      \textgreek{Τίτλος: Αυτόματος Έλεγχος για Σχεδιασμό Γλωσσών} (Random Testing for Language Design)\\
      \textgreek{Επιβλέπων: } Benjamin C. Pierce
    }
    
\edu{\textgreek{Ερευνητική Πρακτική}}{\textgreek{Καλοκαίρι} 2015}
    {Microsoft Research, Cambridge}{
      \textgreek{Πιστοποίηση Νευρωνικών Δικτύων}\\
      \textgreek{Επίβλεψη:} Aditya Nori \textgreek{και} Dimitros Vytiniotis
    }

\edu{\textgreek{Δίπλωμα Ηλεκτρολόγων Μηχανικών και Μηχανικών Ηλεκτρονικών Υπολογιστών}}{2007 - 2012}
    {\textgreek{Εθνικό Μετσόβιο Πολυτεχνείο}}
    {
      \textgreek{Βαθμός}: 9.5/10\\
      \textgreek{Επιβλέπων Διπλωματικής: Κωστής Σαγώνας}
     }

\edu{\textgreek{Απολυτήριο Λυκείου}}{2007}
    {\textgreek{Κολλέγιο Αθηνών}}
    {Magna Cum Laude, 19.6/20}
    
\end{rSection}

%----------------------------------------------------------------------------------------
%	Publications
%----------------------------------------------------------------------------------------

\newcommand{\Catalin}{C\u{a}t\u{a}lin\xspace}
\newcommand{\Hritcu}{Hri\c{t}cu\xspace}
\newcommand{\Denes}{D\'en\`es\xspace}

\newcommand{\pub}[4]{
  {\bf #1} \hfill {\href{https://lemonidas.github.io/pdf/#2}{\bf \em #4}}\\
  {#3}
  }

\newcommand{\pubtwo}[6]{
  {\bf #1} \hfill {\href{https://lemonidas.github.io/pdf/#2}{\bf \em #4}}, {\href{https://lemonidas.github.io/pdf/#5}{\bf \em #6}}\\
  {#3}
  }

\newcommand{\pubsub}[5]{
  {\bf #1:} \hfill {\href{https://lemonidas.github.io/pdf/#2}{\bf \em #4}}\\
  {\bf #5}\\
  {#3}
  }

\begin{rSection}{\textgreek{Βιβλία}}

  {\bf QuickChick: Property-Based Testing in Coq} \hfill {\em 2018}\\
  {\em \textgreek{Λεωνίδας Λαμπρόπουλος και} Benjamin C. Pierce.}\\
  {\textgreek{Σειρά Βιβλίων} ``Software Foundations'', \href{https://softwarefoundations.cis.upenn.edu}{\textgreek{Τόμος} 4}}

  {\bf Luck: A Probabilistic Language for Testing} \hfill {\em \textgreek{κεφάλαιο}}\\
  {\em \textgreek{Λεωνίδας Λαμπρόπουλος}, Diane Gallois-Wong, C\u{a}t\u{a}lin Hri\c{t}cu, John Hughes, Benjamin C. Pierce, \textgreek{και} Li-yao Xia}\\
  \textgreek{Στο βιβλίο}: {Foundations of Probabilistic Programming} \\
  {\em \textgreek{Επιμέλεια}:} {Gilles Barthe, Joost-Pieter Katoen, \textgreek{και} Alexandra Silva}\\

\clearpage
  
\end{rSection}

\begin{rSection}{\textgreek{Δημοσιεύσεις}}

\pub{Coverage Guided, Property Based Testing}
    {FuzzChick.pdf}
    {\textgreek{Λεωνίδας Λαμπρόπουλος}, Michael Hicks, \textgreek{και} Benjamin C. Pierce.}
    {OOPSLA 2019}
    
\pub{Advancing Safety Incrementally with Checked C}
    {CheckedC.pdf}
    {Andrew Ruef, \textgreek{Λεωνίδας Λαμπρόπουλος}, Ian Sweet, David Tarditi, \textgreek{και} Michael Hicks.}
    {POST 2019}

\pub{Keep your Laziness in Check}
    {StrictCheck.pdf}
    {Kenneth Foner, Hengchu Zhang \textgreek{και} \textgreek{Λεωνίδας Λαμπρόπουλος}.}
    {ICFP 2018}

\pub{Generating Good Generators for Inductive Relations}
    {GeneratingGoodGenerators.pdf}
    {\textgreek{Λεωνίδας Λαμπρόπουλος}, Zoe Paraskevopoulou \textgreek{και} Benjamin C.}
    {POPL 2018}

\pub{Ode on a Random Urn (Functional Pearl)}
    {urns.pdf}
    {\textgreek{Λεωνίδας Λαμπρόπουλος}, Antal Spector-Zabusky \textgreek{και} Kenneth Fonner}
    {Haskell 2017}

\pubsub{A Tale of Two Provers}
    {a-tale.pdf}
    {Niiki Vazou, \textgreek{Λεωνίδας Λαμπρόπουλος} \textgreek{και} Jeff Polakow.}
    {Haskell 2017}
    {Verifying Monoidal String Matching in Liquid Haskell and Coq}

\pub{Beginner's Luck: A Language for Property-Based Generators}
    {Luck.pdf}
    {\textgreek{Λεωνίδας Λαμπρόπουλος}, Diane Gallois-Wong, \Catalin \Hritcu, John Hughes, Benjamin C. Pierce, \textgreek{και} Li-yao Xia.}
    {POPL 2017}

\pub{Measuring Neural Net Robustness with Constraints}
    {Robustness.pdf}
    {Osbert Bastani, Yani Ioannou, \textgreek{Λεωνίδας Λαμπρόπουλος}, Dimitrios Vytiniotis, Aditya Nori \textgreek{και} Antonio Criminisi.}
    {NIPS 2016}

\pub{Foundational Property-Based Testing}
    {Foundational.pdf}
    {Zoe Paraskevopoulou, \Catalin \Hritcu, Maxime \Denes, \textgreek{Λεωνίδας Λαμπρόπουλος}, \textgreek{και} Benjamin C. Pierce.}
    {ITP 2015}
    
\pubtwo{Testing Noninterference, Quickly.}
    {TestingNonInterferenceQuickly.pdf}
    {\Catalin \Hritcu, \textgreek{Λεωνίδας Λαμπρόπουλος}, Antal Spector-Zabusky, Arthur
    Azevedo De Amorim, Maxime \Denes, John Hughes, Benjamin C. Pierce, \textgreek{και} Dimitrios
    Vytiniotis.}
    {JFP 2016}
    {TestingNonInterferenceQuickly.pdf}
    {ICFP 2013}
%
%\pub{Testing Noninterference, Quickly.}
%    {TestingNonInterferenceQuickly.pdf}
%    {\Catalin \Hritcu, John Hughes, Benjamin C. Pierce, Antal
%      Spector-Zabusky, Dimitrios Vytiniotis, Arthur Azevedo de Amorim
%      \textgreek{και} \textgreek{Λεωνίδας Λαμπρόπουλος}.}
%    {ICFP 2013}

\pub{Automatic WSDL-guided Test Case Generation\\ for PropEr Testing of Web Services}
    {AutomaticWSDLTesting.pdf}
    {\textgreek{Λεωνίδας Λαμπρόπουλος} \textgreek{και Κωστής Σαγώνας}.}
    {WWV 2012}
   
\end{rSection}

%----------------------------------------------------------------------------------------
%	PATENTS 
%----------------------------------------------------------------------------------------

\begin{rSection}{\textgreek{Πατέντες}}

{\bf Neural Network Image Classifier} \hfill {\href{https://patentimages.storage.googleapis.com/44/ec/9f/3f90c94a602fda/US20170316281A1.pdf}{\bf \em US 2017/0316281}}\\
Antonio Criminisi, Aditya Nori, Dimitrios Vytiniotis, Osbert Bastani, \textgreek{και} \textgreek{Λεωνίδας Λαμπρόπουλος}\\
{\em Microsoft Technology Licensing, LLC}

  
\end{rSection}

%----------------------------------------------------------------------------------------
%	TEACHING 
%----------------------------------------------------------------------------------------

\newcommand{\teach}[3]{
  {#1: #2} \hfill #3 \\
  }

\begin{rSection}{\textgreek{Διδασκαλία}}

  \teach{\textgreek{Καθηγητής}}{Program Analysis and Understanding, CMSC 631, UMD}{\textgreek{Φθινόπωρο} '20}
  \teach{\textgreek{Καθηγητής}}{Program Analysis and Understanding, CMSC 631, UMD}{\textgreek{Φθινόπωρο} '19}
  \teach{\textgreek{Ομιλητής}}{Property-based Random Testing with QuickChick, DeepSpec}{\textgreek{Καλοκαίρι} '18}
  \teach{\textgreek{Ομιλητής}}{Property-based Random Testing with QuickChick, DeepSpec}{\textgreek{Καλοκαίρι} '17}  
  \teach{\textgreek{Βοηθός}}{Advanced Programming, CIS 552, UPenn}{\textgreek{Άνοιξη} '14}
  \teach{\textgreek{Βοηθός}}{Programming Languages and Techniques I, UPenn}{\textgreek{Φθινόπωρο} '13}
  \teach{\textgreek{Βοηθός}}{Programming Techniques, \textgreek{ΕΜΠ}}{\textgreek{Άνοιξη} '12}
  \teach{\textgreek{Βοηθός}}{Intro to Programming, \textgreek{ΕΜΠ}}{\textgreek{Φθινόπωρο} '11}
\end{rSection}


%----------------------------------------------------------------------------------------
%	TALKS
%----------------------------------------------------------------------------------------

\newcommand{\talk}[3]{
  {\bf #1} \\ %\hfill {\href{https://lemonidas.github.io/pdf/#2}{(Slides)}}\\
  {#3}
  }

\newcommand{\apls}[1]{\textgreek{Ετήσιο Σεμινάριο Γλωσσών Προγραμματισμού, Αθήνα, ΕΜΠ, #1}}

\begin{rSection}{\textgreek{Προσκεκλημένες ή Επιμορφωτικές Ομιλίες}}

%\talk{Structured Property-Based Fuzzing}
%     {}
%     {\apls{2019}}

\talk{FuzzChick : Type-Aware Property-Based Fuzzing}
    {}
    {\apls{2019}}
  
\talk{QuickChick : Property-Based Testing in Coq}
    {QuickChickTutorial.pdf}
    {POPL, \textgreek{Lisbon}, 2019}

\talk{StrictCheck : Keep your Laziness in Check}
     {}
    {\apls{2018}}

\talk{Ode to a Random Urn}
     {}
     {\apls{2017}}     

\talk{Random Testing in the Coq Proof Assistant}
     {InvitedCLA.pdf}
     {Keynote, Computational Logic and Applications, Chalmers, 2017}

\talk{Making our Own Luck}
     {}
     {\apls{2016}}

\talk{Making our Own Luck: A Language for Random Generators}
     {}
     {PPS Workshop, 2016}

\talk{Testing Noninterference, Quickly}
     {}
     {\apls{2013}}     
     
\end{rSection}

%----------------------------------------------------------------------------------------
%	SERVICE
%----------------------------------------------------------------------------------------

\newcommand{\serve}[2]{
  {\bf #1:} #2 \\
  }


\begin{rSection}{\textgreek{Υπηρεσία}}

  \serve{ICFP '21}{Workshop co-chair}
  \serve{OOPSLA '22}{ERC member}
  \serve{CLA '20}{PC member}
  \serve{FLOPS '20}{PC member}
  \serve{ICFP '20}{Workshop co-chair}
  \serve{OOPSLA '19}{SRC judge}
  \serve{OOPSLA '19}{PLMW, Panel: PhD Life}
  \serve{PLDI '19}{AEC member}
  \serve{PLAS '19}{PC member}
  \serve{ICFP '18}{PLMW, Panel: Research in Functional Programming}
  \serve{Haskell '16}{External Reviewer}
  
\end{rSection}


%----------------------------------------------------------------------------------------
%	Awards
%----------------------------------------------------------------------------------------

\begin{rSection}{Awards}

  \begin{longtable}{rl}
  2018 & \textgreek{Μεταδιδακτορική Υποτροφία} Victor Basili (UMD)\\
  2014 & \textgreek{Βραβείο Καλύτερης Εκπαιδευτικής Πρακτικής} (Penn Engineering) \\
  2012 & \textgreek{Βραβείο και Υποτροφία από το Ίδρυμα Κρατικών Υποτροφιών (ΙΚΥ)} \\
    & \textgreek{για την πρώτη θέση στο 5ο έτος σπουδών (HMMY)} \\
  \textgreek{Ιούλιος} 2011  & \textgreek{Δεύτερο Βραβείο, 18ος Διεθνής Μαθηματικός Διαγωνισμός} (IMC) \\
  & \textgreek{για φοιτητές Πανεπιστημίου}, Blagoevgrad, Bulgaria \\
  
  2011 & \textgreek{Βραβείο Πάρη Κανελλάκη για την πρώτη θέση στο 3ο και 4ο έτος σπουδών (ΗΜΜΥ)}\\
  2011 & \textgreek{Βραβείο και Υποτροφία από το Ίδρυμα Κρατικών Υποτροφιών (ΙΚΥ)} \\
    & \textgreek{για την πρώτη θέση στο 4ο έτος σπουδών (ΗΜΜΥ)} \\
  \textgreek{Ιούλιος} 2010  & \textgreek{Εύφημη Μνεία, 17ος Διεθνής Μαθηματικός Διαγωνισμός} (IMC) \\
    & \textgreek{για φοιτητές Πανεπιστημίου}, Blagoevgrad, Bulgaria \\

  2010 & \textgreek{Βραβείο και Υποτροφία από το Ίδρυμα Κρατικών Υποτροφιών (ΙΚΥ)} \\
    & \textgreek{για την πρώτη θέση στο 3ο έτος σπουδών (ΗΜΜΥ)} \\
  \textgreek{Ιούλιος} 2009  & \textgreek{Εύφημη Μνεία, 16ος Διεθνής Μαθηματικός Διαγωνισμός} (IMC) \\
    & \textgreek{για φοιτητές Πανεπιστημίου}, Blagoevgrad, Bulgaria \\

  \textgreek{Μάρτιος} 2009  & \textgreek{2ο βραβείο, Μαθηματική Ολυμπιάδα Νοτιοανατολικής Ευρώπης} (SEEMOUS), \textgreek{Κύπρος}\\

  \textgreek{Ιούνιος}  2007 & \textgreek{Απολυτήριο Λυκείου με τιμές} “Magna  Cum  Laude”, \textgreek{Κολλέγιο Αθηνών}\\
             &   \textgreek{Βραβείο Μαθηματικών, Βραβείο Φυσικής, Βραβείο Αριστείας (19.6)}\\
  \textgreek{Φεβρουάριος} 2007 & \textgreek{3ο Βραβείο, 24η Μαθηματική Ολυμπιάδα ``Αρχιμήδης''} \\
  \textgreek{Φεβρουάριος} 2007 & \textgreek{1ο Βραβείο, 67ος Μαθηματικός Διαγωνισμός ``Ευκλείδης''} \\
  2006 &  American Mathematics Competition, \textgreek{Πιστοποιητικό Διάκρισης}\\

  2005-2006 &  \textgreek{Βραβείο Αριστείας (18.6), Κολλέγιο Αθηνών}\\
  \textgreek{Φεβρουάριος} 2006 & \textgreek{1ο Βραβείο, 66ος Μαθηματικός Διαγωνισμός ``Ευκλείδης''} \\  

  2004-2005 &  \textgreek{Βραβείο Αριστείας (19), Κολλέγιο Αθηνών}\\
  \textgreek{Ιούνιος} 2005 & \textgreek{Τρίτο Βραβείο, 9η Βαλκανιάδα Μαθηματικών Νέων} (JBMO), \textgreek{Βέροια}\\
  2004 &  American Mathematics Competition,  \textgreek{Πιστοποιητικό Διάκρισης}\\
  2003-2004 &  \textgreek{Βραβείο Αριστείας (19 και 2 / 13), Κολλέγιο Αθηνών}\\
  2002-2003 &  \textgreek{Βραβείο Αριστείας (19), Κολλέγιο Αθηνών}\\
  2002 & \textgreek{3ο Βραβείο}, American Mathematics Contest (AMC 8)\\
  2001-2002 &  \textgreek{Βραβείο Αριστείας (19 και 8 / 13), Κολλέγιο Αθηνών}\\
%  2001-2002&  Certificate of Distinction\\
  
\end{longtable}

\end{rSection}

%--- LANGUAGES AND SKILLS

\begin{rSection}{\textgreek{Δημοσιευμένα Συστήματα Πληροφορικής}}

  \href{https://github.com/QuickChick/QuickChick}{QuickChick} (OCaml)\\
  \hspace*{2em} \emph{\textgreek{Εργαλέιο Αυτομάτου Ελέγχου μέσω Ιδιοτήτων για} Coq}
  \href {https://github.com/QuickChick/Luck/tree/master/luck}{Luck - Interpreter} (Haskell)\\
  \hspace*{2em} \emph{\textgreek{Ερμηνευτής για τη γλώσσα} Luck}\\
  \href{https://github.com/QuickChick/Luck/tree/master/coq}{Luck - Metatheory} (Coq)\\
  \hspace*{2em} \emph{\textgreek{Μεταθεωρητικές Ιδιότητες της γλώσσας} Luck}\\
  \href{https://github.com/plum-umd/checkedc/tree/master/coq}{Checked C} (Coq)\\
  \hspace*{2em} \emph{\textgreek{Μεταθεωρητικές Ιδιότητες της γλώσσας} Checked C}\\
  \href{https://github.com/microsoft/NeuralNetworkAnalysis}{Neural Network Analysis Framework} (C\#)\\
  \hspace*{2em} \emph{\textgreek{Αφηρημένος Ερμηνευτής Βαθέων Νευρωνικών Δικτύων για Παραγωγή Επιθετικών Παραδειγμάτων}}

\end{rSection}

\begin{rSection}{\textgreek{Φυσικές Γλώσσες}}

  \textgreek{{\bf Ελληνικά}: Μητρική Χρήση}\\
  \textgreek{{\bf Αγγλικά}: Εξαιρετική Χρήση} ({\em Certificate of Proficiency in English, Oxford})\\
  \textgreek{{\bf Γερμανικά}: Βασική Χρήση} ({\em B2 Mittelstufe Deutsch})\\
\end{rSection}



%----------------------------------------------------------------------------------------

\end{document}
