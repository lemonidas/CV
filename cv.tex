%%%%%%%%%%%%%%%%%%%%%%%%%%%%%%%%%%%%%%%%%
% Medium Length Professional CV
% LaTeX Template
% Version 2.0 (8/5/13)
%
% This template has been downloaded from:
% http://www.LaTeXTemplates.com
%
% Original author:
% Trey Hunner (http://www.treyhunner.com/)
%
% Important note:
% This template requires the resume.cls file to be in the same directory as the
% .tex file. The resume.cls file provides the resume style used for structuring the
% document.
%
%%%%%%%%%%%%%%%%%%%%%%%%%%%%%%%%%%%%%%%%%

%----------------------------------------------------------------------------------------
%	PACKAGES AND OTHER DOCUMENT CONFIGURATIONS
%----------------------------------------------------------------------------------------

\documentclass{resume} % Use the custom resume.cls style

\usepackage[left=0.75in,top=0.6in,right=0.75in,bottom=0.6in]{geometry} % Document margins
\usepackage{hyperref}
\usepackage{longtable}
\usepackage{xspace}

\usepackage[utf8]{inputenc}
\usepackage[greek,english]{babel}
\usepackage{lmodern}

\name{Leonidas Lampropoulos} % Your name
%\address{123 Broadway \\ City, State 12345} % Your address
%\address{123 Pleasant Lane \\ City, State 12345} % Your secondary addess (optional)
%\address{(000)~$\cdot$~111~$\cdot$~1111 \\ john@smith.com} % Your phone number and email
\address{Website: \href{https://lemonidas.github.io}{lemonidas.github.io}}
\address{e-mail: leonidas@umd.edu}

\begin{document}

%----------------------------------------------------------------------------------------
%	EDUCATION SECTION
%----------------------------------------------------------------------------------------

\newcommand{\edu}[4]{
  {\bf #3} \hfill {\em #2}\\
  {#1}\\
  #4\\
}

\begin{rSection}{Education and Work Experience}

\edu{Assistant Professor}{July 2020 - present}
    {University of Maryland, College Park}
    {Department of Computer Science\\
     PLUM lab}
  
\edu{Victor Basili Postdoctoral Fellow (Joint Postdoc)}{2018 - 2020}
    {University of Maryland, College Park and University of Pennsylvania}{
      PLUM lab and PLClub - Programming Languages Groups\\
      Supervised by Michael Hicks and Benjamin C. Pierce
    }
% 
%\edu{Postdoc in Computer Science}{2018 - now}
%    {University of Pennsylvania}{
%      Advisor: Benjamin C. Pierce
%    }

\edu{PhD in Computer Science}{2012 - 2018}
    {University of Pennsylvania}{
%      GPA: 3.99\\
      Title: Random Testing for Language Design\\
      Advisor: Benjamin C. Pierce
    }
    
\edu{Research Internship}{Summer 2015}
    {Microsoft Research, Cambridge}{
      Formal Methods for Neural Network Verification\\
      Advisors: Aditya Nori and Dimitros Vytiniotis
    }

\edu{Diploma in Electrical Engineering and Computer Science}{2007 - 2012}
    {National Technical University of Athens}
    {
      Thesis Advisor: Kostis Sagonas
     }

%\edu{High School Diploma}{2007}
%    {Athens College}
%    {Athens, Greece}
%    {Magna Cum Laude, 19.6/20}
%    
\end{rSection}

%----------------------------------------------------------------------------------------
%	Grants
%----------------------------------------------------------------------------------------

\begin{rSection}{Funding}

  {\bf NSF:SHF:CAREER:}{Fuzzing Formal Specifications}\hfill{582.352\$}\\
  2022-2027\\
  
  {\bf NSF:SHF:Medium:}{Efficient and Trustworthy Proof Engineering}\hfill{539.956\$}\\
  {\em with Milos Gligoric.} 2021-2024\\
  
  {\bf NSF:SHF:Medium:}Bringing Python up to Speed \hfill{374.390\$}\\
  {\em with Emery Berger, Michael Hicks, and Benjamin C. Pierce.} 2020-2023\\
  
\end{rSection}

%----------------------------------------------------------------------------------------
%	Publications
%----------------------------------------------------------------------------------------

\newcommand{\Catalin}{C\u{a}t\u{a}lin\xspace}
\newcommand{\Hritcu}{Hri\c{t}cu\xspace}
\newcommand{\Denes}{D\'en\`es\xspace}

\newcommand{\pub}[4]{
  {\bf #1} \hfill {\href{https://lemonidas.github.io/pdf/#2}{\bf \em #4}}\\
  {#3}
  }

\newcommand{\pubtwo}[6]{
  {\bf #1} \hfill {\href{https://lemonidas.github.io/pdf/#2}{\bf \em #4}}, {\href{https://lemonidas.github.io/pdf/#5}{\bf \em #6}}\\
  {#3}
  }

\newcommand{\pubsub}[5]{
  {\bf #1:} \hfill {\href{https://lemonidas.github.io/pdf/#2}{\bf \em #4}}\\
  {\bf #5}\\
  {#3}
  }

\begin{rSection}{Books}

  {\bf QuickChick: Property-Based Testing in Coq} \hfill {\em 2018}\\
  {\em Leonidas Lampropoulos and Benjamin C. Pierce.}\\
  {Software Foundation Series, \href{https://softwarefoundations.cis.upenn.edu}{Volume 4}}

  {\bf Luck: A Probabilistic Language for Testing} \hfill {\em chapter, to appear}\\
  {\em Leonidas Lampropoulos, Diane Gallois-Wong, C\u{a}t\u{a}lin Hri\c{t}cu, John Hughes, Benjamin C. Pierce, and Li-yao Xia} {(editors: Gilles Barthe, Joost-Pieter Katoen, and Alexandra Silva)}\\
  In book: {Foundations of Probabilistic Programming} \\

  
\end{rSection}

\begin{rSection}{Publications}

\pub{Computing Correctly with Inductive Relations}
    {ComputingCorrectly.pdf}
    {Zoe Paraskevopoulou, Aaron Eline, and Leonidas Lampropoulos.}
    {PLDI 2022}
  
\pub{A Formal Model for Checked C}
    {CheckedCFormal.pdf}
    {Liyi Li, Yiyun Liu, Deena Postol, Leonidas Lampropoulos, David Van Horn, and Michael Hicks.}
    {CSF 2022}

\pubsub{Do Judge a Test by its Cover}
    {Combinatorial.pdf}
    {Harrison Goldstein, John Hughes, Leonidas Lampropoulos, and Benjamin C. Pierce.}
    {ESOP 2021}
    {Combining Combinatorial and Property-Based Testing}    
    
\pub{Coverage Guided, Property Based Testing}
    {FuzzChick.pdf}
    {Leonidas Lampropoulos, Michael Hicks, and Benjamin C. Pierce.}
    {OOPSLA 2019}
    
\pub{Advancing Safety Incrementally with Checked C}
    {CheckedC.pdf}
    {Andrew Ruef, Leonidas Lampropoulos, Ian Sweet, David Tarditi, and Michael Hicks.}
    {POST 2019}

\pub{Keep your Laziness in Check}
    {StrictCheck.pdf}
    {Kenneth Foner, Hengchu Zhang and Leonidas Lampropoulos.}
    {ICFP 2018}

\pub{Generating Good Generators for Inductive Relations}
    {GeneratingGoodGenerators.pdf}
    {Leonidas Lampropoulos, Zoe Paraskevopoulou and Benjamin C.}
    {POPL 2018}

\pub{Ode on a Random Urn (Functional Pearl)}
    {urns.pdf}
    {Leonidas Lampropoulos, Antal Spector-Zabusky and Kenneth Fonner}
    {Haskell 2017}

\pubsub{A Tale of Two Provers}
    {a-tale.pdf}
    {Niiki Vazou, Leonidas Lampropoulos and Jeff Polakow.}
    {Haskell 2017}
    {Verifying Monoidal String Matching in Liquid Haskell and Coq}

\pub{Beginner's Luck: A Language for Property-Based Generators}
    {Luck.pdf}
    {Leonidas Lampropoulos, Diane Gallois-Wong, \Catalin \Hritcu, John Hughes, Benjamin C. Pierce, and Li-yao Xia.}
    {POPL 2017}

\pub{Measuring Neural Net Robustness with Constraints}
    {Robustness.pdf}
    {Osbert Bastani, Yani Ioannou, Leonidas Lampropoulos, Dimitrios Vytiniotis, Aditya Nori and Antonio Criminisi.}
    {NIPS 2016}

\pub{Foundational Property-Based Testing}
    {Foundational.pdf}
    {Zoe Paraskevopoulou, \Catalin \Hritcu, Maxime \Denes, Leonidas Lampropoulos, and Benjamin C. Pierce.}
    {ITP 2015}
    
\pubtwo{Testing Noninterference, Quickly.}
    {TestingNonInterferenceQuickly.pdf}
    {\Catalin \Hritcu, Leonidas Lampropoulos, Antal Spector-Zabusky, Arthur
    Azevedo De Amorim, Maxime \Denes, John Hughes, Benjamin C. Pierce, and Dimitrios
    Vytiniotis.}
    {JFP 2016}
    {TestingNonInterferenceQuickly.pdf}
    {ICFP 2013}
%
%\pub{Testing Noninterference, Quickly.}
%    {TestingNonInterferenceQuickly.pdf}
%    {\Catalin \Hritcu, John Hughes, Benjamin C. Pierce, Antal
%      Spector-Zabusky, Dimitrios Vytiniotis, Arthur Azevedo de Amorim
%      and Leonidas Lampropoulos.}
%    {ICFP 2013}

\pub{Automatic WSDL-guided Test Case Generation\\ for PropEr Testing of Web Services}
    {AutomaticWSDLTesting.pdf}
    {Leonidas Lampropoulos and Kostis Sagonas.}
    {WWV 2012}
   
\end{rSection}

\clearpage

%----------------------------------------------------------------------------------------
%	PATENTS 
%----------------------------------------------------------------------------------------

\begin{rSection}{Patents}

{\bf Neural Network Image Classifier} \hfill {\href{https://patentimages.storage.googleapis.com/44/ec/9f/3f90c94a602fda/US20170316281A1.pdf}{\bf \em US 2017/0316281}}\\
Antonio Criminisi, Aditya Nori, Dimitrios Vytiniotis, Osbert Bastani, and Leonidas Lampropoulos\\
{\em Microsoft Technology Licensing, LLC}

  
\end{rSection}

%----------------------------------------------------------------------------------------
%	TEACHING 
%----------------------------------------------------------------------------------------

\newcommand{\teach}[3]{
  {#1: #2} \hfill #3 \\
  }

\begin{rSection}{Teaching}

  \teach{Instructor}{Design and Implementation of Programming Languages, CMSC 430, UMD}{Spring '21}  
  \teach{Instructor}{Program Analysis and Understanding, CMSC 631, UMD}{Fall '20}
  \teach{Instructor}{Program Analysis and Understanding, CMSC 631, UMD}{Fall '19}
  \teach{Lecturer}{Property-based Random Testing with QuickChick, DeepSpec Summer School}{Summer '18}
  \teach{Lecturer}{Property-based Random Testing with QuickChick, DeepSpec Summer School}{Summer '17}  
%  \teach{TA}{Advanced Programming, CIS 552, UPenn}{Spring '14}
%  \teach{TA}{Programming Languages and Techniques I, UPenn}{Fall '13}
%  \teach{TA}{Programming Techniques, NTUA}{Spring '12}
%  \teach{TA}{Intro to Programming, NTUA}{Fall '11}
\end{rSection}


%----------------------------------------------------------------------------------------
%	TALKS
%----------------------------------------------------------------------------------------

\newcommand{\talk}[3]{
  {\bf #1} \\ %\hfill {\href{https://lemonidas.github.io/pdf/#2}{(Slides)}}\\
  {#3}
  }

\newcommand{\apls}[1]{Athens PL Seminar, NTUA, #1}

\begin{rSection}{Invited Talks and Tutorials}

\talk{Adventures in Property-Based Testing}
     {}
     {Research Challenges in Computer Science, NTUA, 2022}

\talk{Computing Correctly with Inductive Relations}
     {}
     {Athens PL Seminar, NTUA, 2021}

\talk{Property-Based Testing for OCaml through Coq}     
     {}
     {OCaml Workshop, ICFP, 2021}

\talk{Do Judge a Test by it's Cover}
     {}
     {Athens PL Seminar, NTUA, 2020}
     
\talk{Adventures in Property-Based Testing}
     {}
     {UIC PL Seminar, Spring 2020}
  
\talk{Software Correctness through Testing and Verification}
     {}
     {IMDEA (Invited Talk), 2020}
  
\talk{Structured Property-Based Fuzzing}
     {}
     {Athens PL Seminar, NTUA, 2019}

\talk{FuzzChick : Type-Aware Property-Based Fuzzing}
    {}
    {\apls{2019}}
  
\talk{QuickChick : Property-Based Testing in Coq}
    {QuickChickTutorial.pdf}
    {POPL TutorialFest, Lisbon, 2019}

\talk{StrictCheck : Keep your Laziness in Check}
     {}
     {Athens PL Seminar, NTUA, 2018}

\talk{Ode to a Random Urn}
     {}
     {Athens PL Seminar, NTUA, 2017}

\talk{Random Testing in the Coq Proof Assistant}
     {InvitedCLA.pdf}
     {Keynote, Computational Logic and Applications, Chalmers, 2017}

\talk{Making our Own Luck}
     {}
     {Athens PL Seminar, NTUA, 2016}

\talk{Making our Own Luck: A Language for Random Generators}
     {}
     {PPS Workshop, 2016}

\talk{Testing Noninterference, Quickly}
     {}
     {Athens PL Seminar, NTUA, 2013}
     
\end{rSection}

%----------------------------------------------------------------------------------------
%	SERVICE
%----------------------------------------------------------------------------------------

\newcommand{\serve}[2]{
  {\bf #1:} #2 \\
  }


\begin{rSection}{Service}

  \serve{CoqPL '22}{PC Member}
  \serve{POPL '22}{PC Member}
  \serve{ICFP '21}{PC Member}
  \serve{ICFP '21}{Workshop co-chair}
  \serve{OOPSLA '22}{ERC member}
  \serve{CLA '20}{PC member}
  \serve{FLOPS '20}{PC member}
  \serve{ICFP '20}{Workshop co-chair}
  \serve{OOPSLA '19}{SRC judge}
  \serve{OOPSLA '19}{PLMW, Panel: PhD Life}
  \serve{PLDI '19}{AEC member}
  \serve{PLAS '19}{PC member}
  \serve{ICFP '18}{PLMW, Panel: Research in Functional Programming}
  \serve{Haskell '16}{External Reviewer}
  
\end{rSection}

%----------------------------------------------------------------------------------------
%	Awards
%----------------------------------------------------------------------------------------

\begin{rSection}{Awards}

  \begin{longtable}{rl}
  2022 & NSF CAREER Award \\
  2018 & Victor Basili PostDoctoral Fellowship (UMD)\\
  2014	 & Teaching Practicum Award (Penn Engineering) \\
  2012 & State Scholarship Foundation (IKY) award and scholarship \\
       & for first place during the 5th year of studies \\
  % Placeholder : Pari Kanellaki
  July  2011  &  Second  Prize,  18th  International Mathematical Competition (IMC) \\
  &  for University Students , Blagoevgrad, Bulgaria \\
  2011 & IKY first place award and scholarship (4th year)\\
  July 2010 &  Honorable Mention,  17th IMC, Blagoevgrad, Bulgaria \\
  2010 & IKY first place award and scholarship (3rd year)\\
  July 2009 &  Honorable Mention,  16th IMC, Budapest, Hungary\\
  Mar 2009  &  2nd Award, South Eastern European Mathematical Olympiad, Cyprus\\

  June  2007 &   High  School  Diploma with  honors  “Magna  Cum  Laude”\\
             &   Mathematics Award, Physics Award,  Excellence Award (19.6)\\
  Feb 2007 &  3rd Award, 24th Mathematical Olympiad “Archimedes”\\
  Feb 2007 &  1st Award, 67th Panhellenic Math Contest “Euclid”\\
  2006 &  American Mathematics Competition, Certificate of Distinction\\
  2005-2006 &  Excellence Award (18.6)\\
  Feb 2006 &  1st Award, 66th Panhellenic Math Contest “Euclid”\\
  2004-2005 &  Excellence Award (19)\\
  June 2005 &  Third Prize, 9th Junior Balcan Mathematical Olympiad, Veroia\\
  2004 &  American Mathematics Competition, Certificate of Distinction\\
  2003-2004&  Excellence Award (19 and 2/13)\\
  2002-2003&  Excellence Award (19)\\
  2002 & American Mathematics Contest (AMC 8), third place\\
  2001-2002&  Excellence Award (19 and 8/13)\\
%  2001-2002&  Certificate of Distinction\\
  
\end{longtable}

\end{rSection}

%--- LANGUAGES AND SKILLS

\begin{rSection}{Projects}

  \href{https://github.com/QuickChick/QuickChick}{QuickChick}\\
  \hspace*{2em} \emph{Property-Based Testing Tool for Coq (OCaml)}\\
  \href {https://github.com/QuickChick/Luck/tree/master/luck}{Luck - Interpreter}\\
  \hspace*{2em} \emph{Interpreter for Luck (Haskell)}\\
  \href{https://github.com/QuickChick/Luck/tree/master/coq}{Luck - Metatheory}\\
  \hspace*{2em} \emph{Metatheory for Luck (Coq)}\\
  \href{https://github.com/plum-umd/checkedc/tree/master/coq}{Checked C}\\
  \hspace*{2em} \emph{Metatheory for Checked C (Coq)}\\
  \href{https://github.com/microsoft/NeuralNetworkAnalysis}{Neural Network Analysis Framework}\\
  \hspace*{2em} \emph{Abstract Interpreter for Deep Neural Networks for Generating Adversarial Examples (C\#)}

\end{rSection}

\begin{rSection}{Languages}

  {\bf Greek}: Native \\
  {\bf English}: Excellent ({\em Certificate of Proficiency in English, Oxford})\\
  {\bf German}: Basic ({\em B2 Mittelstufe Deutsch})\\
\end{rSection}



%----------------------------------------------------------------------------------------

\end{document}
