%%%%%%%%%%%%%%%%%%%%%%%%%%%%%%%%%%%%%%%%%
% Medium Length Professional CV
% LaTeX Template
% Version 2.0 (8/5/13)
%
% This template has been downloaded from:
% http://www.LaTeXTemplates.com
%
% Original author:
% Trey Hunner (http://www.treyhunner.com/)
%
% Important note:
% This template requires the resume.cls file to be in the same directory as the
% .tex file. The resume.cls file provides the resume style used for structuring the
% document.
%
%%%%%%%%%%%%%%%%%%%%%%%%%%%%%%%%%%%%%%%%%

%----------------------------------------------------------------------------------------
%	PACKAGES AND OTHER DOCUMENT CONFIGURATIONS
%----------------------------------------------------------------------------------------

\documentclass{resume} % Use the custom resume.cls style

\usepackage[left=0.75in,top=0.6in,right=0.75in,bottom=0.6in]{geometry} % Document margins
\usepackage{hyperref}

\name{Leonidas Lampropoulos} % Your name
%\address{123 Broadway \\ City, State 12345} % Your address
%\address{123 Pleasant Lane \\ City, State 12345} % Your secondary addess (optional)
%\address{(000)~$\cdot$~111~$\cdot$~1111 \\ john@smith.com} % Your phone number and email
\address{Website: \href{https://lemonidas.github.io}{lemonidas.github.io}}
\address{e-mail: leonidaslamp@hotmail.com} 

\begin{document}

%----------------------------------------------------------------------------------------
%	EDUCATION SECTION
%----------------------------------------------------------------------------------------

\newcommand{\edu}[4]{
  {\bf #3} \hfill {\em #2}\\
  {#1}\\
  #4\\
}

\begin{rSection}{Education and Work Experience}

\edu{PostDoc in Computer Science}{2018 - now}
    {University of Maryland, College Park}{
      PLUM lab - Programming Languages Group\\
      Advisor: Michael Hicks
    }

\edu{PostDoc in Computer Science}{2018 - now}
    {University of Pennsylvania}{
      PLClub - Programming Languages Group\\
      Advisor: Benjamin C. Pierce
    }

\edu{PhD in Computer Science}{2012 - 2018}
    {University of Pennsylvania}{
%      GPA: 3.99\\
      Title: Random Testing for Language Design\\
      Advisor: Benjamin C. Pierce
    }
    
\edu{Research Internship}{Summer 2015}
    {Microsoft Research, Cambridge}{
      Formal Methods for Neural Network Verification\\
      Advisors: Aditya Nori and Dimitros Vytiniotis
    }

\edu{Diploma in Electrical Engineering and Computer Science}{2007 - 2012}
    {National Technical University of Athens}
    {
      GPA: 9.5/10\\
      Thesis Advisor: Kostis Sagonas
     }

\edu{High School Diploma}{2007}
    {Athens College}
    {Athens, Greece}
    {Magna Cum Laude, 19.6/20}
    
\end{rSection}

%----------------------------------------------------------------------------------------
%	Publications
%----------------------------------------------------------------------------------------

\newcommand{\pub}[4]{
  {\bf #1} \hfill {\href{https://lemonidas.github.io/pdf/#2}{\bf \em #4}}\\
  {#3}
  }

\newcommand{\pubsub}[5]{
  {\bf #1:} \hfill {\href{https://lemonidas.github.io/pdf/#2}{\bf \em #4}}\\
  {\bf #5}\\
  {#3}
  }

\begin{rSection}{Books}
  {\bf QuickChick: Property-Based Testing in Coq} \hfill {\em 2018}\\
  {\em Leonidas Lampropoulos and Benjamin C. Pierce.}
  {Software Foundation Series, \href{https://softwarefoundations.cis.upenn.edu}{Volume 4}}

\end{rSection}

\begin{rSection}{Publications}

\pub{Coverage Guided, Property Based Testing}
    {FuzzChick.pdf}
    {Leonidas Lampropoulos, Michael Hicks, and Benjamin C. Pierce.}
    {OOPSLA 2019}
    
\pub{Advancing Safety Incrementally with Checked C}
    {CheckedC.pdf}
    {Andrew Ruef, Leonidas Lampropoulos, Ian Sweet, David Tarditi, and Michael Hicks.}
    {POST 2019}


\pub{Keep your Laziness in Check}
    {StrictCheck.pdf}
    {Kenneth Foner, Hengchu Zhang and Leonidas Lampropoulos.}
    {ICFP 2018}

\pub{Generating Good Generators for Inductive Relations}
    {GeneratingGoodGenerators.pdf}
    {Leonidas Lampropoulos, Zoe Paraskevopoulou and Benjamin C.}
    {POPL 2018}

\pub{Ode on a Random Urn (Functional Pearl)}
    {urns.pdf}
    {Leonidas Lampropoulos, Antal Spector-Zabusky and Kenneth Fonner}
    {Haskell Symposium 2017}


\pubsub{A Tale of Two Provers}
    {a-tale.pdf}
    {Niiki Vazou, Leonidas Lampropoulos and Jeff Polakow.}
    {Haskell Symposium 2017}
    {Verifying Monoidal String Matching in Liquid Haskell and Coq}

\pub{Beginner's Luck: A Language for Property-Based Generators}
    {Luck.pdf}
    {Leonidas Lampropoulos, Diane Gallois-Wong, Cătălin Hriţcu, John Hughes, Benjamin C. Pierce, and Li-yao Xia.}
    {POPL 2017}


\pub{Measuring Neural Net Robustness with Constraints}
    {Robustness.pdf}
    {Osbert Bastani, Yani Ioannou, Leonidas Lampropoulos, Dimitrios Vytiniotis, Aditya Nori and Antonio Criminisi.}
    {NIPS 2016}


\pub{Foundational Property-Based Testing}
    {Foundational.pdf}
    {Zoe Paraskevopoulou, Cătălin Hriţcu, Maxime Dénès, Leonidas Lampropoulos, and Benjamin C. Pierce.}
    {ITP 2015}

\pub{Testing Noninterference, Quickly.}
    {TestingNonInterferenceQuickly.pdf}
    {Cătălin Hriţcu, Leonidas Lampropoulos, Antal Spector-Zabusky, Arthur
    Azevedo De Amorim, Maxime Dénès, John Hughes, Benjamin C. Pierce, and Dimitrios
    Vytiniotis.}
    {Journal of Functional Programming 2016}

\pub{Testing Noninterference, Quickly.}
    {TestingNonInterferenceQuickly.pdf}
    {Cătălin Hriţcu, John Hughes, Benjamin C. Pierce, Antal
      Spector-Zabusky, Dimitrios Vytiniotis, Arthur Azevedo de Amorim
      and Leonidas Lampropoulos.}
    {ICFP 2013}

\pub{Automatic WSDL-guided Test Case Generation for PropEr Testing of Web Services}
    {AutomaticWSDLTesting.pdf}
    {Leonidas Lampropoulos and Kostis Sagonas.}
    {WWV 2012}
   
\end{rSection}

%----------------------------------------------------------------------------------------
%	TALKS
%----------------------------------------------------------------------------------------

\newcommand{\talk}[3]{
  {\bf #1} \\ %\hfill {\href{https://lemonidas.github.io/pdf/#2}{(Slides)}}\\
  {#3}
  }


\begin{rSection}{Invited Talks and Tutorials}

\talk{QuickChick : Property-Based Testing in Coq}
    {QuickChickTutorial.pdf}
    {POPL TutorialFest, Lisbon, 2019}

\talk{Random Testing in the Coq Proof Assistant}
     {InvitedCLA.pdf}
     {Computational Logic and Applications, Chalmers, 2017}

\end{rSection}


%----------------------------------------------------------------------------------------
%	EXAMPLE SECTION
%----------------------------------------------------------------------------------------

%\begin{rSection}{Section Name}

%Section content\ldots

%\end{rSection}

%----------------------------------------------------------------------------------------

\end{document}
